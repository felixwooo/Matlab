\documentclass[a4paper,12pt]{article} %make sure that we are using 12 pt font
\usepackage{fancyhdr}
\usepackage[margin=2.5cm]{geometry}%rounded up from 1.87, just to be safe...
\usepackage{parskip}
%%\usepackage{times} %make sure that the times new roman is used
\usepackage{sectsty}
%%\raggedright



\begin{document}


\pagestyle{fancy}
%This puts your name at the top right, outside the margin
\lhead{CPSC 607: Biological Computation (Winter 2016)\\Project Documentation}
\rhead{Samiul Azam\\30 Apr. 2016}
 \renewcommand{\headrulewidth}{1pt}
\sectionfont{\fontsize{12}{15}\selectfont}
\vspace{2cm}


\section*{\\Phase 02: Tuning Each Classifier}
............................................................................................................................................

In the command window of Matlab, type ``help $<$\verb|file_name.m|$>$'' to see the header documentation of my program scripts.\\
............................................................................................................................................

\textbf{Note 1:}\\\\
This folder contains following main script. Go to Matlab Command Window and type the target main script (For example $>>$  \verb|tune_classifier|) for execution.
\begin{itemize}
	\item \verb|tune_classifier.m|   
		\begin{itemize}
		\item Read the main script header for more details.
		\item This program will tune (maximize the accuracy) each classifier iteratively by changing one of their sensitive parameter.
		
		\item It generates following four ``txt'' files and four ``bmp'' files (in the current directory) that contain graphs and information about the peak point of a graph.	
				\begin{itemize}
					\item \verb|disa_tune.bmp|
					\item \verb|svm_tune.bmp|
					\item \verb|knn_tune.bmp|
					\item \verb|tree_tune.bmp|
					\item \verb|disa_output.txt|
					\item \verb|svm_output.txt|
					\item \verb|knn_output.txt|
					\item \verb|tree_output.txt|	\\\\			
				\end{itemize}
		\end{itemize}
\end{itemize}

............................................................................................................................................\\\\
\textbf{Note 2: }\\
This folder contains following function scripts (You don't need to run them separately).
\begin{itemize}
	
	\item \verb|disa_tune.m|
	\begin{itemize}
		\item Read the function header for more details.
		\item This function is used inside \verb|tune_classifier.m|.
	\end{itemize}
	\item \verb|tree_tune.m|
	\begin{itemize}
		\item Read the function header for more details.
		\item This function is used inside \verb|tune_classifier.m|.
	\end{itemize}
	\item \verb|knn_tune.m|
	\begin{itemize}
		\item Read the function header for more details.
		\item This function is used inside \verb|tune_classifier.m|.
	\end{itemize}
	\item \verb|svm_tune.m|
	\begin{itemize}
		\item Read the function header for more details.
		\item This function is used inside \verb|tune_classifier.m|.
	\end{itemize}
	\item \verb|my_svmclassify.m|
	\begin{itemize}
		\item Not my major contribution. I slightly modify the built-in matlab function to generate probability instead of `0' or `1' labels. 
		\item This function is used inside \verb|tune_classifier.m|.
	\end{itemize}
	\item \verb|my_svmdecision.m|
	\begin{itemize}
		\item Not my major contribution. I slightly modify the built-in matlab function to generate probability instead of `0' or `1' labels. 
		\item This function is used inside \verb|my_svmclassify.m|.
	\end{itemize}
	
\end{itemize}
............................................................................................................................................\\
In the command window of Matlab, type ``help $<$\verb|file_name.m|$>$'' to see the header documentation of my program scripts.\\
............................................................................................................................................

\end{document}